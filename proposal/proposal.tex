%%%%%%%%%%%%%%%%%%%%%%%%%%%%%%%%%%%%%%%%%%%%%%%%%%%%%%%%%%%%%%%%%%%%%%%%%%%%%%%%%%%%%%%%%%%%%%%%
%
% CSCI 1290 Writeup Template
%
%%%%%%%%%%%%%%%%%%%%%%%%%%%%%%%%%%%%%%%%%%%%%%%%%%%%%%%%%%%%%%%%%%%%%%%%%%%%%%%%%%%%%%%%%%%%%%%%

\documentclass[11pt]{article}

\usepackage[english]{babel}
\usepackage[utf8]{inputenc}
\usepackage[colorlinks = true,
            linkcolor = blue,
            urlcolor  = blue]{hyperref}
\usepackage[a4paper,margin=0.75in]{geometry}
\usepackage{stackengine,graphicx}
\usepackage{fancyhdr}
\setlength{\headheight}{15pt}
\usepackage{microtype}
\usepackage{times}
\usepackage{booktabs}
%\usepackage{hyperref} % for links
\usepackage{subcaption} % for subfigures
\usepackage{enumitem} % extra package for no spacing between lists
\usepackage{amsmath} % extra package for aligned equations
\usepackage{listings} % for code blocks

% From https://ctan.org/pkg/matlab-prettifier
%\usepackage[numbered,framed]{matlab-prettifier}

\frenchspacing
\setlength{\parindent}{0cm} % Default is 15pt.
\setlength{\parskip}{0.3cm plus1mm minus1mm}

\pagestyle{fancy}
\fancyhf{}
\lhead{Final Project Proposal}
\rhead{CSCI 1290}
\rfoot{\thepage}

% objects for maketitle
\author{}
\date{2020.11.22}
\title{\vspace{-1cm}Object-tracked Fourier Blur OR CNN image upscale}

\begin{document}
\maketitle
\vspace{-1cm}
\thispagestyle{fancy}

\section*{Project Proposal Summary - Proposal one}

We learned in class that a fourier transform can either add a blur or remove a blur
from a specific depth in an image. I have previous experience with convolution with images to find
objects. What I propose is multi-step process that places a local blur around a target.
This will occur based on the following steps

\begin{itemize}
 \item Apply a convolution of an image and a kernel that finds the position of the object we wish to blur, as well as it's depth
 \item use the depth information to know what fourier transform to use to apply a blur at that depth
 \item merge the original image and the blurred image so that only a bounding box around the object from the blurred
   image is replaced in the original image
\end{itemize}

Depending on the kernel used, this can allow blurring of small parts of an image while retaining clear
picture elsewhere for example in privacy centric applications such as peoples faces, or documents with
personal information. The only thing I'm uncertain is how well we will be able to find the correct depth
and testing of fourier transforms to target that specific depth.

A reach goal would be to make it robust in being able to work in reverse as well, so take an unknown
image with a local blur and deblur it and check that the output image is sharp everywhere.

\section*{Project Proposal Summary - Proposal two}

Another potential project is one that uses neural networks to upscale video. this would take high quality 4k video,
downscale and add noise to either 540p or 1080p, run through a neural network that takes these low scale images and
over multiple layers upscales the image back to the original resolution, with loss based on how well the
upscaled image matches the original. This project should be quicker to implement than the first, yet how it performs
is likely more suspect without delving into deeper networks with more fine-tuning.


%%%%%%%%%%%%%%%%%%%%%%%%%%%%%%%%%%%

% clear page for images to position where desired
\clearpage

%%%%%%%%%%%%%%%%%%%%%%%%%%%%%%%%%%%

\end{document}

